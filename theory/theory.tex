\chapter{Theory}
\section{The history of wakefield acceleration}
Tajima Dawson great Idee, MTV bubble regime, same time PWFA von Rosenzweig... first linear measurements for PWFA
\section{Plasma physics}
An introduction into plasma physics is given by starting with chens definition. Topics are
\subsection{debye shielding}
\subsection{time scales}
scattering can be talked about

\subsection{plasma definition}
There are different types of plasmas, but we are only handling with thin, cold  weakly coupled plasmas
\subsection{electromagnetic waves in plasmas}
dispersion relation needs to be talked about. Especially the dispersion relation of lasers during ionization should get some insight here. One has to look into ionization defocussing.  
\subsection{fluid model of plasmas}
\subsection{waves in plasmas}
So far everything still flows nicely with working along Chen and Mulser. However now the turn needs to be taken.
The reason is wavebreaking
\subsection{wavebreaking}
Wavebreaking gives us the the ideal way to go from plasma description to blowout description.
\section{PWFA theory}
\subsection{history of PWFA}
A short historic overview is given. 
Maybe mention landau damping? Then of course, Tajima,Dawson. Also MTV and Rosenzweig should be mentioned.

\section{The blowout regime}

\section{Descriptions for the blowout regime}
Lotov, Suk, breakdown of fluid theory
Q-tilde and resonant wake excitation.
\subsection{Trapping conditions}
Basic calculations for the trapping potential are shown.
The particle movement must be solved in 3D and in 1D. Make a picture of the comparisons.


\section{Accelerator physics}
Acc. physics should clearly be introduced. Emittance, brightness, twiss parameter need to be defined. Floettmann. A good book should be used here. Don't know which one,yet. TBD.
\subsection{Panowsky-Wenzel Theorem}
 \begin{equation}
W_r =\partial_r W_z
\end{equation} 
This theorem is so important, it clearly needs a subsection. But where ?
\section{laser ionisation description}
(see diss by Ihar Shchatsinin FU Berlin)
\subsection{Keldysh Parameter}

With $E_{bind}$ being the binding energy and $U_p=\frac{q^2 I}{2 m_e \epsilon_0 c \omega^2}$ being the ponderomotive energy 
\begin{equation}
\gamma=\sqrt{\frac{E_{bind}}{2U_p}}
\end{equation}

$\gamma >1 $ -> Multiphoton Ionisation\\
$\gamma < 1$ tunnel ionisation or BSI

\subsection{ADK theory}
Tunnel ionization is great