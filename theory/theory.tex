%\documentclass{thesis}
%\bibliographystyle{apsrev4-1}
%\usepackage[latin1]{inputenc}% erm\"oglich die direkte Eingabe der Umlaute 
%\usepackage[T1]{fontenc} % das Trennen der Umlaute
%\usepackage{cite}
%\usepackage{graphicx}
%\usepackage{subfig}
%\usepackage{caption}
%\usepackage{amssymb,amsmath}
%%\def\BibTeX{{\rm B\kern-.05em{\sc i\kern-.025em b}\kern-.08em
% %   T\kern-.1667em\lower.7ex\hbox{E}\kern-.125emX}}
%\usepackage[english,english]{babel}   

%\chapter{Theory}
\section{The history of wakefield acceleration}
Tajima Dawson great Idee, MTV bubble regime, same time PWFA von Rosenzweig... first linear measurements for PWFA
\section{Plasma physics}
An introduction into plasma physics is given by starting with chens definition. Topics are
\subsection{debye shielding}
\subsection{time scales}
scattering can be talked about

\subsection{plasma definition}
There are different types of plasmas, but we are only handling with thin, cold  weakly coupled plasmas
\subsection{electromagnetic waves in plasmas}
dispersion relation needs to be talked about. Especially the dispersion relation of lasers during ionization should get some insight here. One has to look into ionization defocussing.  
\subsection{fluid model of plasmas}
\subsection{waves in plasmas}
So far everything still flows nicely with working along Chen and Mulser. However now the turn needs to be taken.
The reason is wavebreaking
\subsection{wavebreaking}
Wavebreaking gives us the the ideal way to go from plasma description to blowout description.
\section{PWFA theory}
\subsection{history of PWFA}
A short historic overview is given. 
Maybe mention landau damping? Then of course, Tajima,Dawson. Also MTV and Rosenzweig should be mentioned.

\section{The blowout regime}

\section{Descriptions for the blowout regime}
Lotov, Suk, breakdown of fluid theory
Q-tilde and resonant wake excitation.


\section{Accelerator physics}
Acc. physics should clearly be introduced. Emittance, brightness, twiss parameter need to be defined. Floettmann. A good book should be used here. Don't know which one,yet. TBD.
\subsection{Panowsky-Wenzel Theorem}
 \begin{equation}
W_r =\partial_r W_z
\end{equation} 
This theorem is so important, it clearly needs a subsection. But where ?


\section{Electron Trapping in plasma accelerators}
%The blowout regime inherits an excellent wake field distribution to efficiently accelerate electrons due to the large accelerating fields of the order of $10^{10} \mathrm{V/m}$ and the strong focusing fields in the back of the bubble. Energy can easily be transferred from an driver bunch to a trailing witness electron bunch, which has been sucessfully shown in two-bunch experiments at SLAC. A chirped electron bunch was accelerated to 42 GeV in the linear accelerator, rotated in a magnetic chicane, where the central part of the electron beam was cut out by a collimator and rotated back so that two seperate electron bunches where trailing on the same orbit. In the subsequent plasma stage the wakefields accelerated the witness bunch to up to 84 GeV in only ??? m of acceleration. This external injection scheme was very successfull and has been further studied during the entire experimental period of FACET. With such high fields 
%Even though the fields are DAMN

In order to derive an expression for the trapping condition of a single electron in PWFA, one has to start with the equation of motion for such a single electron. 
\begin{equation}
F=\frac{d\vec{p}}{dt}=q(\vec{E}\times \vec{B})
\end{equation}
with the electron charge $q$ electric field $\vec{E}$ and magnetic field $\vec{B}$

This leads to the single particle electron hamiltonian $ H=\gamma m c^2+\Phi$ with the temporal derivative.
\begin{align}
\frac{dH}{dt}&=\frac{d}{dt} (\gamma m_e c^2)+\frac{d}{dt}(q\Phi)\\
&=\vec{v}\frac{d\vec{p}}{dt}+\frac{d}{dt}(q\Phi)\\
&=q\vec{v}(-\nabla \Phi-\frac{\partial \vec{A}}{\partial t})+\frac{\vec{v}\times\vec{B}}{c}+\frac{d}{dt}(q\Phi)\\
&=q(\frac{d}{dt}\Phi-\vec{v}\vec{\nabla}\Phi-\vec{v}\frac{\partial \vec{A}}{\partial t})\\
&=q(\frac{\partial \Phi}{\partial t}-\vec{v}\frac{\partial \vec{A}}{\partial t})
\end{align}

If one assumes now, that the wake fields are constant during the trapping process, then 

\begin{equation}
(\frac{\partial}{\partial t}+v_\mathrm{\phi} \frac{\partial}{\partial z} ) f =   f ( z-v_\mathrm{\phi} t)
\end{equation}\begin{equation}
(\frac{\partial}{\partial t}+v_\mathrm{\phi} \frac{\partial}{\partial z} ) f =0 \ \forall \   f (\vec{r}, z-v_\mathrm{\phi} t)
\end{equation}
which is especially true for the hamiltonian.
\begin{align*}
\frac{d}{dt}H&=q(\frac{\partial \Phi}{\partial t}-\vec{v}\frac{\partial \vec{A}}{\partial t})\\
&=-q v_\mathrm{\phi}(\frac{\partial \Phi}{\partial z}-\vec{v} \frac{\partial \vec{A}}{\partial z}) 
\end{align*}

Since $H-v_\mathrm{\phi}P_z=\mathrm{const.}$

\begin{align}
H-v_\mathrm{\phi}P_z &= \mathrm{const.}\\
\gamma m c^2+\Psi-v_\mathrm{\phi}p_z-v_\mathrm{\phi}qA_z &= \mathrm{const.}\\
\gamma+\frac{q \Phi}{m c^2}-v_\mathrm{\phi} \frac{p_z}{mc^2} &= \mathrm{const.}\\
\gamma - v_\mathrm{\phi} \frac{p_z}{mc^2}- \underbrace{\frac{q}{mc^2}(\Phi-v_\mathrm{\phi}A_z)}_{\bar{\Psi}}  &= \mathrm{const.} 
\end{align}
$\bar{\Psi}$ is the trapping potential, that determines the potential difference for an electron in a potential that moves with a phase velocity $v_\mathrm{\phi}$ with respect to the laboratory frame. It is valid for small as for relativistic velocities.
With the trapping potential one can calculate if an electron inside the plasma wake will be accelerated or not i.e. if electrons will be able to catch up with the wake's velocity during the propagation of the wake or if it will slip out of the potential.
From the prior calculations a general formula can be determined, that compares

\begin{equation}
\label{eq:Trapping_Potential_Pre}
\Delta \bar{\Psi}= \bar{\Psi}_\mathrm{i}-\bar{\Psi}_\mathrm{f}=\gamma_\mathrm{f}-\gamma_\mathrm{i}-\gamma_f\frac{v_\mathrm{\phi}v_\mathrm{f}}{c^2}+\gamma_\mathrm{i}\frac{v_\mathrm{\phi}v_\mathrm{i}}{c^2} 
\end{equation}

In order to honor the name "trapping potential" i.e. to apply this derivation to make predictions of the electron trapping behavior in the plasma wake, it is necessary to define a trapping condition. 
An obvious and conventional choice is that an electron should catch up with the wake's velocity so that 
$v_\mathrm{f}=v_\mathrm{\phi}$.
Equation \ref{eq:Trapping_Potential_Pre} consequently simplifies to 
\begin{equation}
\label{eq:Trapping_Potential_Raw}
\Delta \bar{\Psi}= \gamma_\mathrm{\phi}-\gamma_\mathrm{i}-\gamma_\mathrm{\phi}\frac{v_\mathrm{\phi}^2}{c^2}+\gamma_\mathrm{i}\frac{v_\mathrm{\phi}v_\mathrm{i}}{c^2} 
\end{equation}
Equation \ref{eq:Trapping_Potential_Raw} can be further separated into different physical cases:
\subsection{luminal wakefield, electron injected  at rest}
In this case the plasma wake travels with a phase velocity near the speed of light, which is the case for beam-driven scenarios with high $\gamma$ driver beams ($v_\mathrm{\phi} \approx c$), and electrons starting inside the wake initially at rest ($v_\mathrm{i} \approx \ 0$).
Here Equation \ref{eq:Trapping_Potential_Raw} simplifies to
\begin{equation}
\Delta \bar{\Psi}=-1
\end{equation}
Examples of this case are the underdense photocathode, or Trojan Horse injection\cite{Hidding_PRL_2012}, or wakefield induced ionization injection\cite{MartinezdelaOssa2014231}.
\subsection{subluminal wakefield, electron injected at rest}

\begin{equation}
\label{eq:Trapping_Potential_DTH}
\Delta \bar{\Psi}=\gamma_\mathrm{\phi}(1-\frac{v_\mathrm{\phi}^2}{c^2})-1=\gamma_\mathrm{\phi}^{-1}-1
\end{equation}
This formula can for example be applied to ionization injection in LWFA\cite{PakPRL2012} or beam-driven ionization injection schemes in which the wake's phase velocity is retarded such as the Downramp-assisted Trojan Horse (DTH)\cite{DTH}, which this work has as special focus on. In latter case mathematically strictly speaking $\frac{d H}{dt}\neq0 $, but for small changes $\frac{dH}{dt}\approx 0$ during the injection process of the electrons, equation \ref{eq:Trapping_Potential_DTH} can still be applied.
\subsection{subluminal wakefield, electrons at rest}

\subsection{superluminal wakefield}
There are physical situations imaginable in which the wake or at least part of the wake move with a phase velocity faster than the speed of light. This is the case for example when a beam driven wake traverses an electron density upramp.
From the previous deductions it seems obvious, that trapping electrons in such a superluminal wakefield is not possible, as $\gamma_\mathrm{\phi}^{-1}$ becomes complex for $v_\mathrm{\phi}>c$.

However, if the superluminosity is only transient as with a short density upramp, the phase velocity will return to $c$ right after the transition. In this case trpping can be possible, but the mathematic tool presented in this section is insufficient to describe the trapping and the phase velocity after the transition is setting the demand on the potential.
\subsection{From the Trapping Potential to the Current Profile - velocity bunching at it's max.}
\subsubsection{emittance preservation}
why don't you ... look at space charge effect to estimate required gamma/acceleration and focusing forces for emittance preservation ? Try looking at Pak's thesis and this emittance paper that claims one needs a fast acceleration, but knows nothing about Trojan Horse yet.

\section{laser ionisation description}
(see diss by Ihar Shchatsinin FU Berlin)

\subsection{Keldysh Parameter}

With $E_{bind}$ being the binding energy and $U_p=\frac{q^2 I}{2 m_e \epsilon_0 c \omega^2}$ being the ponderomotive energy 
\begin{equation}
\gamma=\sqrt{\frac{E_{bind}}{2U_p}}
\end{equation}

$\gamma >1 $ -> Multiphoton Ionisation\\
$\gamma < 1$ tunnel ionisation or BSI

\subsection{ADK theory}
Tunnel ionization is great
	
	
\section{Influence of the ionization behaviour}
	Here it is to be described how for TH the co-moving frame is advantagous. dependence of ionization front to 
	create witness beams.
	Image of Ionization front.
	
	
	\subsection{Rake injection}
	It could be discussed here how a witness generation might be different for the wake electric fields.
	For example, there is no movement of the release position in xi. But the release won't happen in the potential minimum.
\section{Trojan Horse}
\label{sec:Theory_TrojanHorse}
	
\section{Downramp assisted Trojan Horse}
	
