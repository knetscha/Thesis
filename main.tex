\documentclass{thesis}
%\usepackage[latin1]{inputenc}% erm\"oglich die direkte Eingabe der Umlaute 
%\usepackage[T1]{fontenc} % das Trennen der Umlaute
%\usepackage{cite}
%\usepackage{graphicx}
%\usepackage{subfig}
%\usepackage{caption}
%\usepackage{standalone}% Need standalone package
%\usepackage{amssymb,amsmath}
%\usepackage{pdfpages}
\usepackage{dsfont}
\usepackage{graphicx}
\usepackage{placeins}
\usepackage{amsmath}
\usepackage{cite}
\usepackage{wasysym}
\usepackage{longtable}
\usepackage{siunitx}
\usepackage{wrapfig}
\usepackage{geometry}
\usepackage{color}
\usepackage{hyperref}
\usepackage{caption}
\usepackage{enumitem}
\usepackage{gensymb}
\usepackage[english,english]{babel}  

\newcommand{\mE}{m_\mathrm{e}} 
\newcommand{\nE}{n_\mathrm{e}} 
\newcommand{\qE}{q_\mathrm{e}}
\newcommand{\wP}{\omega_\mathrm{p}}
\newcommand{\lP}{\lambda_\mathrm{p}}
\newcommand{\kP}{k_\mathrm{p}}
\title{Trapping and Beamloading in hybrid Plasma Wakefield Accelerator schemes}
\author{Alexander Knetsch}

\date{\today}
\begin{document}
%sds \cite{Oz_PhD}
\maketitle
\tableofcontents
\chapter{theory}
\chapter{Theory}
\section{The history of wakefield acceleration}
Tajima Dawson great Idee, MTV bubble regime, same time PWFA von Rosenzweig... first linear measurements for PWFA
\section{Plasma physics}
An introduction into plasma physics is given by starting with chens definition. Topics are
\subsection{debye shielding}
\subsection{time scales}
scattering can be talked about

\subsection{plasma definition}
There are different types of plasmas, but we are only handling with thin, cold  weakly coupled plasmas
\subsection{electromagnetic waves in plasmas}
dispersion relation needs to be talked about. Especially the dispersion relation of lasers during ionization should get some insight here. One has to look into ionization defocussing.  
\subsection{fluid model of plasmas}
\subsection{waves in plasmas}
So far everything still flows nicely with working along Chen and Mulser. However now the turn needs to be taken.
The reason is wavebreaking
\subsection{wavebreaking}
Wavebreaking gives us the the ideal way to go from plasma description to blowout description.
\section{PWFA theory}
\subsection{history of PWFA}
A short historic overview is given. 
Maybe mention landau damping? Then of course, Tajima,Dawson. Also MTV and Rosenzweig should be mentioned.

\section{The blowout regime}

\section{Descriptions for the blowout regime}
Lotov, Suk, breakdown of fluid theory
Q-tilde and resonant wake excitation.
\subsection{Trapping conditions}
Basic calculations for the trapping potential are shown.
The particle movement must be solved in 3D and in 1D. Make a picture of the comparisons.


\section{Accelerator physics}
Acc. physics should clearly be introduced. Emittance, brightness, twiss parameter need to be defined. Floettmann. A good book should be used here. Don't know which one,yet. TBD.
\subsection{Panowsky-Wenzel Theorem}
 \begin{equation}
W_r =\partial_r W_z
\end{equation} 
This theorem is so important, it clearly needs a subsection. But where ?


\subsection{Trapping in PWFA}
%The blowout regime inherits an excellent wake field distribution to efficiently accelerate electrons due to the large accelerating fields of the order of $10^{10} \mathrm{V/m}$ and the strong focusing fields in the back of the bubble. Energy can easily be transferred from an driver bunch to a trailing witness electron bunch, which has been sucessfully shown in two-bunch experiments at SLAC. A chirped electron bunch was accelerated to 42 GeV in the linear accelerator, rotated in a magnetic chicane, where the central part of the electron beam was cut out by a collimator and rotated back so that two seperate electron bunches where trailing on the same orbit. In the subsequent plasma stage the wakefields accelerated the witness bunch to up to 84 GeV in only ??? m of acceleration. This external injection scheme was very successfull and has been further studied during the entire experimental period of FACET. With such high fields 
%Even though the fields are DAMN

In order to derive an expression for the trapping condition of a single electron in PWFA, one has to start with the equation of motion for such a single electron. 
\begin{equation}
F=\frac{d\vec{p}}{dt}=q(\vec{E}\times \vec{B})
\end{equation}
with the electron charge $q$ electric field $\vec{E}$ and magnetic field $\vec{B}$

This leads to the single particle electron hamiltonian $ H=\gamma m c^2+\Phi$ with the temporal derivative.
\begin{align}
\frac{dH}{dt}&=\frac{d}{dt} (\gamma m_e c^2)+\frac{d}{dt}(q\Phi)\\
&=\vec{v}\frac{d\vec{p}}{dt}+\frac{d}{dt}(q\Phi)\\
&=q\vec{v}(-\nabla \Phi-\frac{\partial \vec{A}}{\partial t})+\frac{\vec{v}\times\vec{B}}{c}+\frac{d}{dt}(q\Phi)\\
&=q(\frac{d}{dt}\Phi-\vec{v}\vec{\nabla}\Phi-\vec{v}\frac{\partial \vec{A}}{\partial t})\\
&=q(\frac{\partial \Phi}{\partial t}-\vec{v}\frac{\partial \vec{A}}{\partial t})
\end{align}

If one assumes now, that the wake fields are constant during the trapping process, then 

\begin{equation}
(\frac{\partial}{\partial t}+v_\mathrm{\phi} \frac{\partial}{\partial z} ) f =   f ( z-v_\mathrm{\phi} t)
\end{equation}\begin{equation}
(\frac{\partial}{\partial t}+v_\mathrm{\phi} \frac{\partial}{\partial z} ) f =0 \ \forall \   f (\vec{r}, z-v_\mathrm{\phi} t)
\end{equation}
which is especially true for the hamiltonian.
\begin{align*}
\frac{d}{dt}H&=q(\frac{\partial \Phi}{\partial t}-\vec{v}\frac{\partial \vec{A}}{\partial t})\\
&=-q v_\mathrm{\phi}(\frac{\partial \Phi}{\partial z}-\vec{v} \frac{\partial \vec{A}}{\partial z}) 
\end{align*}

Since $H-v_\mathrm{\phi}P_z=\mathrm{const.}$

\begin{align}
H-v_\mathrm{\phi}P_z &= \mathrm{const.}\\
\gamma m c^2+\Psi-v_\mathrm{\phi}p_z-v_\mathrm{\phi}qA_z &= \mathrm{const.}\\
\gamma+\frac{q \Phi}{m c^2}-v_\mathrm{\phi} \frac{p_z}{mc^2} &= \mathrm{const.}\\
\gamma - v_\mathrm{\phi} \frac{p_z}{mc^2}- \underbrace{\frac{q}{mc^2}(\Phi-v_\mathrm{\phi}A_z)}_{\bar{\Psi}}  &= \mathrm{const.} 
\end{align}
$\bar{\Psi}$ is the trapping potential, that determines the potential difference for an electron in a potential that moves with a phase velocity $v_\mathrm{\phi}$ with respect to the laboratory frame. It is valid for small as for relativistic velocities.
With the trapping potential one can calculate if an electron inside the plasma wake will be accelerated or not i.e. if electrons will be able to catch up with the wake's velocity during the propagation of the wake or if it will slip out of the potential.
From the prior calculations a general formula can be determined, that compares

\begin{equation}
\label{eq:Trapping_Potential_Pre}
\Delta \bar{\Psi}= \bar{\Psi}_\mathrm{i}-\bar{\Psi}_\mathrm{f}=\gamma_\mathrm{f}-\gamma_\mathrm{i}-\gamma_f\frac{v_\mathrm{\phi}v_\mathrm{f}}{c^2}+\gamma_\mathrm{i}\frac{v_\mathrm{\phi}v_\mathrm{i}}{c^2} 
\end{equation}

In order to honor the name "trapping potential" i.e. to apply this derivation to make predictions of the electron trapping behavior in the plasma wake, it is necessary to define a trapping condition. 
An obvious and conventional choice is that an electron should catch up with the wake's velocity so that 
$v_\mathrm{f}=v_\mathrm{\phi}$.
Equation \ref{eq:Trapping_Potential_Pre} consequently simplifies to 
\begin{equation}
\label{eq:Trapping_Potential_Raw}
\Delta \bar{\Psi}= \gamma_\mathrm{\phi}-\gamma_\mathrm{i}-\gamma_\mathrm{\phi}\frac{v_\mathrm{\phi}^2}{c^2}+\gamma_\mathrm{i}\frac{v_\mathrm{\phi}v_\mathrm{i}}{c^2} 
\end{equation}
Equation \ref{eq:Trapping_Potential_Raw} can be further separated into different physical cases:
\subsubsection{luminal wakefield, electron injected  at rest}
In this case the plasma wake travels with a phase velocity near the speed of light, which is the case for beam-driven scenarios with high $\gamma$ driver beams ($v_\mathrm{\phi} \approx c$), and electrons starting inside the wake initially at rest ($v_\mathrm{i} \approx \ 0$).
Here Equation \ref{eq:Trapping_Potential_Raw} simplifies to
\begin{equation}
\Delta \bar{\Psi}=-1
\end{equation}
Examples of this case are the underdense photocathode, or Trojan Horse injection, or wakefield induced ionization injection.
\subsubsection{sublimunal wakefield, electron injected at rest}
\section{laser ionisation description}
(see diss by Ihar Shchatsinin FU Berlin)
	
\subsection{Keldysh Parameter}

With $E_{bind}$ being the binding energy and $U_p=\frac{q^2 I}{2 m_e \epsilon_0 c \omega^2}$ being the ponderomotive energy 
\begin{equation}
\gamma=\sqrt{\frac{E_{bind}}{2U_p}}
\end{equation}

$\gamma >1 $ -> Multiphoton Ionisation\\
$\gamma < 1$ tunnel ionisation or BSI

\subsection{ADK theory}
Tunnel ionization is great
%\chapter{Simulations}
\section{Start-to-End simulations for a Trojan Horse at FlashForward experiment}
\section{The trapping potential}
\section{plasma density profile optimisation}
\subsection{Density Downramp facilitated Trojan Horse Acceleration}

In %\cite{Fub_PRL_downramp}
 wird beschrieben, wie sich ein negativer Dichtegradient auf die in einer Plasmawelle getrappten Elektronen auswirkt.
Ausgangspunkt ist dabei die LWFA mit der Annahme, dass bei konstater Elektronendichte der Laserpuls, sowie die Bubble sich mit c bewegen.\\
Die Welle bewege sich in z-Richtung.\\
Die L\"ange der bubble ist Abh\"angig von der Plasmadichte $n$. Da sich diese \"uber eine downramp von $n_i$ auf $n_f$ verringert, vergr\"o\ss ert sich auch die bubble \"uber diese Strecke.
Sei $$\xi=z-ct$$ die Position relativ zum Laserpuls. $\xi<0$ beschreibt die Position eines Elektrons in der Bubble hinter dem Laserpuls.
Das Elektron bleibt am gleichen Ort relativ zur Bubblestruktur, aber die Bubble ver\"andert sich. Die Phase der Bubble ver\"andert sich von 
$$\Psi_i=\frac{\omega_{pi}}{c}\xi$$ nach
$$\Psi_f=\frac{\omega_{pf}}{c}\xi$$
mit der Phasendifferenz:
$$\Delta\Psi=\Psi_f-\Psi_i=\Psi_i[1-(\frac{n_f}{n_i})^{1/2}]$$
Die entsprechende Phasengeschwindigkeit errechnet sich aus 
$$v_p=-\frac{\frac{\partial \Psi}{\partial(ct)}}{\frac{\partial \Psi}{\partial z}}=\frac{c}{1+\frac{1}{2\omega_p(z)n(z)}\frac{\partial n(z)}{\partial z}\xi }$$

$$\Delta \lambda_\mathrm{p} \propto (e^{\frac{1}{2}C_\mathrm{ramp}z_1}-e^{\frac{1}{2}C_\mathrm{ramp}z_2}) $$
$$\approx \frac{1}{2}C_\mathrm{ramp}(z_1-z_2)  $$
\subsection{density transition injection suppression}
See paper by Suk

\section{Laser Beam shaping for optimisation}
\subsection{my beamloading description}
\subsection{beamloading in theory}

\subsection{beamshaping in reality}
-pulse-shaping by spatial light modulator(SLM) (Meshulach adaptive real-time fs pulse shaping)

\section{Magnetic Field facilitated Trojan Horse Acceleration}


\chapter{experiment}
%\chapter{Experiment}
\section{FACET}
\subsection{The FACET experimental setup}

\section{FlashForwad}
\section{Clara}



%\chapter{Experiment}
\section{FACET}
\subsection{The FACET experimental setup}

\section{FlashForwad}
\section{Clara}


\bibliographystyle{unsrt}
\bibliography{lib}
%\bibliographystyle{plain}    
\end{document}