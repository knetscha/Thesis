\documentclass{thesis}
% Damit die Verwendung der deutschen Sprache nicht ganz so umst\"andlich wird,
% sollte man die folgenden Pakete einbinden: 
\usepackage[latin1]{inputenc}% erm\"oglich die direkte Eingabe der Umlaute 
\usepackage[T1]{fontenc} % das Trennen der Umlaute
\usepackage{cite}
\usepackage{graphicx}
\usepackage{subfig}
\usepackage{caption}
\usepackage{amssymb,amsmath}
%\def\BibTeX{{\rm B\kern-.05em{\sc i\kern-.025em b}\kern-.08em
 %   T\kern-.1667em\lower.7ex\hbox{E}\kern-.125emX}}
\usepackage[german,german]{babel} % hiermit werden deutsche Bezeichnungen genutzt und 
                     % die W\"orter werden anhand der neue Rechtschreibung 
		     % automatisch getrennt.  
		     
		     
		     
		     \newcommand{\mE}{m_\mathrm{e}} 
\newcommand{\nE}{n_\mathrm{e}} 
\newcommand{\qE}{q_\mathrm{e}}
\newcommand{\wP}{\omega_\mathrm{p}}
\newcommand{\lP}{\lambda_\mathrm{p}}
\newcommand{\kP}{k_\mathrm{p}}
\title{Trapping and Beamloading in hybrid Plasma Wakefield Accelerator schemes}
\author{Alexander Knetsch}

\date{\today}
% Hinweis: \title{um was auch immer es geht}, \author{wer es auch immer 
% geschrieben hat} und  \date{wann auch immer das war} k\"onnen vor 
% oder nach dem  Kommando \begin{document} stehen 
% Aber der \maketitle Befehl mu\ss{} nach dem \begin{document} Kommando stehen! 
% TEst3
\begin{document}

\maketitle
\tableofcontents


As mentioned before the phase velocity of the plasma wake in PWFA is equal to the velocity of the driving electron bunch $v_\phi=v_\mathrm{bunch}$ in the case of a constant plasma density, which is normally close to the speed of light. This feature of PWFA makes dephasing, which is the effect of the witness beam approaching the drive beam in the co-moving frame due to velocity differences between the bunches, mostly negligible. However, it also makes the injection of electrons into the wake particularly challenging.
Since the plasma wavelength $\lP$ (eq. , which is the characteristic scale of the longitudinal blowout length, depends on the plasma density the blowout must expand and contract on a density downramp or upramp respectively.

The phase position of the wake is
\begin{equation}
\phi=\kP\xi \propto \sqrt{\nE}\xi
\end{equation}
in the co-moving frame with $\xi=z-ct$.

The phase velocity is 



\begin{align}
v_\phi(\xi,z)&=-c\frac{\frac{\partial \phi}{\partial (ct)}}{\frac{\partial \phi}{\partial z}}\\
&=\frac{(\frac{\partial \kP}{\partial t}\xi+\frac{\partial \xi}{\partial t} \kP)}{\frac{\partial \kP}{\partial z}\xi + \frac{\partial x}{\partial z}\kP}\\
&=\frac{c\kP}{\kP\frac{1}{2\nE}\frac{ \partial \nE}{\partial z}\xi+\kP }\\
&=\frac{c}{\frac{1}{2\nE}\frac{ \partial \nE}{\partial z}\xi+1 }
\end{align}





\begin{equation}
\frac{E_r-B_\theta}{E_0}=\frac{k_p}{2}r
\end{equation}

\begin{equation}
\frac{\partial E_z}{\partial r}=\frac{\partial E_r}{\partial z} \approx 0
\end{equation}









If one assumes now, that the wake fields are constant during the trapping process, then 

\begin{align}
H-v_\mathrm{\phi}P_z &= \mathrm{const.}\\
\gamma m c^2+\Psi-v_\mathrm{\phi}p_z-v_\mathrm{\phi}qA_z &= \mathrm{const.}\\
\gamma+\frac{q \Phi}{m c^2}-v_\mathrm{\phi} \frac{p_z}{mc^2} &= \mathrm{const.}\\
\gamma - v_\mathrm{\phi} \frac{p_z}{mc^2} \underbrace{\frac{q}{mc^2}(\Phi-v_\mathrm{\phi}A_z)}_{\hat{\Psi}}  &= \mathrm{const.} 
\end{align}
%- \underbrace{\frac{q}{mc^2}(\Phi-v_\mathrm{\phi}A_z)}_

which is especially true for the hamiltonian.
\begin{align*}
\frac{d}{dt}H&=q(\frac{\partial \Phi}{\partial t}-\vec{v}\frac{\partial \vec{A}}{\partial t})\\
&=-q v_\mathrm{\phi}(\frac{\partial \Phi}{\partial z}-\vec{v} \frac{\partial \vec{A}}{\partial z})
\end{align*}
The trapping of a single electron in PWFA happens in a short time (i.e. a short propagation distance) compared to the timescales on which the wakefield changes its shape. An example for a distance over which the wakefield is modified is the betatron length FORUMLA !!!.
This gives the convenient possibility to treat the problem in a frame moving along with the phase velocity of the wake $v_\phi$. Mathematically this can be done by finding a constant $C_\mathrm{H}$ with $\frac{d C_\mathrm{H}}{dt}=0$ , so that $\frac{d}{dt}(H-C_\mathrm{H})=0$.
W. Lu suggested in his thesis [citation needed !!!] 
\begin{align*}
\frac{d}{dt}(H-v_\mathrm{\phi} P_z)&=-qv_\mathrm{\phi}(\frac{\partial \Phi}{\partial z}-\vec{v}\frac{\partial \vec{A}}{\partial z})-qv_\mathrm{\phi}(v_z \frac{\partial A_z}{\partial z}-\frac{\partial \Phi}{\partial z})\\
&\approx q v_\mathrm{\phi}(v_z \frac{\partial A_z}{\partial z}-v_z \frac{\partial A_z}{\partial z})=0
\end{align*}

\begin{align*}
H-v_\mathrm{\phi}P_z&=const.\\
\gamma m c^2+q\Phi-v_\mathrm{\phi}p_z-v_\mathrm{\phi}qA_z&=const.\\
\gamma+\frac{q\Phi}{mc^2}-v_\mathrm{\phi}\frac{p_z}{mc^2}-v_\mathrm{\phi}q\frac{A_z}{mc^2}&=const\\
\gamma-v_\mathrm{\phi}\frac{p_z}{mc^2}-\underbrace{\frac{q}{mc^2}(\Phi-v_\mathrm{\phi}A_z)}_{\Psi}&=const.\\
-const. +\gamma + v_\mathrm{\phi}\frac{p_z}{mc^2}&=\Psi
\end{align*}

\begin{align*}
\Psi_f-\Psi_i=\gamma_f-\gamma_i-v_\mathrm{\phi}\frac{\gamma_f m v_f}{mc^2}
\end{align*}

\begin{align*}
Q'(z)=\int_{-\infty}^{\infty}1- exp(W_\mathrm{ADK}(z,t))\ dt
\end{align*}


\bibliography{lib}
\bibliographystyle{plain}    
\end{document}