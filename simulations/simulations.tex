\chapter{Simulations}
\section{Start-to-End simulations for a Trojan Horse at FlashForward experiment}
\section{The trapping potential}
\section{plasma density profile optimisation}
\subsection{Density Downramp facilitated Trojan Horse Acceleration}

In %\cite{Fub_PRL_downramp}
 wird beschrieben, wie sich ein negativer Dichtegradient auf die in einer Plasmawelle getrappten Elektronen auswirkt.
Ausgangspunkt ist dabei die LWFA mit der Annahme, dass bei konstater Elektronendichte der Laserpuls, sowie die Bubble sich mit c bewegen.\\
Die Welle bewege sich in z-Richtung.\\
Die L\"ange der bubble ist Abh\"angig von der Plasmadichte $n$. Da sich diese \"uber eine downramp von $n_i$ auf $n_f$ verringert, vergr\"o\ss ert sich auch die bubble \"uber diese Strecke.
Sei $$\xi=z-ct$$ die Position relativ zum Laserpuls. $\xi<0$ beschreibt die Position eines Elektrons in der Bubble hinter dem Laserpuls.
Das Elektron bleibt am gleichen Ort relativ zur Bubblestruktur, aber die Bubble ver\"andert sich. Die Phase der Bubble ver\"andert sich von 
$$\Psi_i=\frac{\omega_{pi}}{c}\xi$$ nach
$$\Psi_f=\frac{\omega_{pf}}{c}\xi$$
mit der Phasendifferenz:
$$\Delta\Psi=\Psi_f-\Psi_i=\Psi_i[1-(\frac{n_f}{n_i})^{1/2}]$$
Die entsprechende Phasengeschwindigkeit errechnet sich aus 
$$v_p=-\frac{\frac{\partial \Psi}{\partial(ct)}}{\frac{\partial \Psi}{\partial z}}=\frac{c}{1+\frac{1}{2\omega_p(z)n(z)}\frac{\partial n(z)}{\partial z}\xi }$$

$$\Delta \lambda_\mathrm{p} \propto (e^{\frac{1}{2}C_\mathrm{ramp}z_1}-e^{\frac{1}{2}C_\mathrm{ramp}z_2}) $$
$$\approx \frac{1}{2}C_\mathrm{ramp}(z_1-z_2)  $$
\subsection{density transition injection suppression}
See paper by Suk

\section{Laser Beam shaping for optimisation}
\subsection{my beamloading description}
\subsection{beamloading in theory}

\subsection{beamshaping in reality}
-pulse-shaping by spatial light modulator(SLM) (Meshulach adaptive real-time fs pulse shaping)

\section{Magnetic Field facilitated Trojan Horse Acceleration}

